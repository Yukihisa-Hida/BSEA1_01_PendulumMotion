\documentclass{jarticle}

%このようにパーセントを打つと,その行はコメントになります.メモ書きに使えますよ!

\title{振子の運動}
\author{2511198 肥田幸久}
\date{2025年5月2日}

\begin{document}
\maketitle

\section{目的}

本実験では、棒振子の角度のデータから、角度の時間変化の様子、振子の周期
や摩擦の大きさ等を求め、運動を解析する。

\section{原理}

振子の振れ幅を$\theta(t)$, 棒の長さを$L$, 棒の重さを$m$, 粘性摩擦係数を$b$とする
と, 運動方程式は次式になる.

\begin{equation}
  \frac{3}{1}mL^2\frac{d^2\theta}{dt^2}+b\frac{d\theta}{dt}+mg\frac{L}{2}sin\theta=0
\end{equation}

整理すると

\begin{equation}
  \frac{d^2\theta}{dt^2}+\frac{3b}{mL^2}\frac{d\theta}{dt}+\frac{3g}{2L}sin\theta=0
\end{equation}

ここで

\begin{itemize}
  \item $\alpha=\frac{3b}{mL^2}$:減衰係数
  \item $\omega_0=\sqrt{\frac{3g}{2L}}$:非減衰時の自然振動数(大振幅でもおおよその指標になる)
\end{itemize}


\section{方法}

金属棒の端を角度検出センサに取り付け, 棒が振動した際の時間とともに変化する角度に対応したデータ(電圧値)をマイコンで読み取る.
マイコン内に保存されたデータをパソコンに保存し測定データとする.

電圧はセンサの抵抗値が$0\Omega$のとき$0V$, $10k\Omega$のとき電源電圧である$3V$となり, それ
ぞれ角度が$0°$と$360°$に対応している. この電圧値は$10$bit AD変換後の値として記
録されるため, マイコンに記録された角度データの値を$s$とすれば,

\begin{equation}
  \theta=360°\times\frac{s}{2^{10}-1}
\end{equation}

により角度を求めることができる.

\section{結果}

\section{考察}


\end{document}

